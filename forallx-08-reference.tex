%!TEX root = forallxsyr.tex
\chapter[Quick Reference]{Quick Reference}
%\pagestyle{plain}
\section*{Truth Functional Operators}


\begin{center}
\begin{tabular}{c c|c|c|c|c|c}
\meta{A} & \meta{B} & $\enot \meta{A}$ & $\meta{A}\eand\meta{B}$ & $\meta{A}\eor\meta{B}$ & $\meta{A}\eif\meta{B}$ & $\meta{A}\eiff\meta{B}$\\
\hline
T & T & F & T & T & T & T\\
T & F & F & F & T & F & F\\
F & T & T & F & T & T & F\\
F & F & T & F & F & T & T
\end{tabular}
\end{center}



%\section{Symbolisation}
%\begin{center}
%\label{app.symbolization}
%\begin{tabular*}{\textwidth}{rl}
%\multicolumn{2}{c}{\textsc{Sentential Connectives}}\\ \\
%It is not the case that P & $\enot P$\\
%Either P, or Q & $(P \eor Q)$\\
%Neither P, nor Q & $\enot(P \eor Q)$\ or \ $(\enot P \eand \enot Q)$\\
%Both P, and Q & $(P \eand Q)$\\
%If P, then Q & $(P \eif Q)$\\
%P only if Q & $(P \eif Q)$\\
%P if and only if Q & $(P \eiff Q)$\\
%P unless Q & $(P \eor Q)$\\
%\\
%\multicolumn{2}{c}{\label{SymbolizingPredicates}\textsc{Predicates}}\\ \\
%All Fs are Gs & $\forall x(Fx \eif Gx)$\\
%Some Fs are Gs & $\exists x(Fx \eand Gx)$\\
%Not all Fs are Gs & $\enot\forall x(Fx \eif Gx)$\ or\ $\exists x(Fx \eand \enot Gx)$\\
%No Fs are Gs & $\forall x(Fx \eif\enot Gx)$\ or\ $\enot\exists x(Fx \eand Gx)$\\
%\\
%\multicolumn{2}{c}{\textsc{Identity}}\\ \\
%Only c is G & $\forall x(Gx \eiff x=c)$\\
%Everything besides c is G & $\forall x(\enot x = c \eif Gx)$\\
%%$j$ is more $R$ than anyone else. & $\forall x(x\neq j \eif Rjx)$\\
%The F is G & $\exists x(Fx \eand \forall y(Fy \eif x=y) \eand Gx)$\\
%It is not the case that the F is G & $\enot\exists x(Fx \eand \forall y(Fy \eif x=y) \eand Gx)$\\
%The F is non-G & $\exists x(Fx \eand \forall y(Fy \eif x=y) \eand \enot Gx)$
%\end{tabular*}
%\end{center}
%
%
%
%
%
%
%% BEGIN: symbolizing cardinality
%
%\newpage
%\section{Using identity to symbolize quantities}
%
%\subsection*{There are at least \blank\ Fs.}
%\label{summary.atleast}
%
%\begin{ekey}
%\item[\text{one}] $\exists xFx$
%\item[\text{two}] $\exists x_1\exists x_2(Fx_1 \eand Fx_2 \eand \enot x_1  = x_2)$
%\item[\text{three}] $\exists x_1\exists x_2\exists x_3(Fx_1 \eand Fx_2 \eand Fx_3 \eand \enot x_1 = x_2 \eand\enot x_1 = x_3 \eand \enot x_2 = x_3)$
%\item[\text{four}] $\exists x_1\exists x_2\exists x_3\exists x_4 (Fx_1 \eand Fx_2 \eand Fx_3 \eand Fx_4 \eand \phantom{x}$\\
%\phantom{$\exists x_1\exists x_2$}$\enot x_1 = x_2 \eand \enot x_1 = x_3 \eand \enot x_1 = x_4 \eand \enot x_2 = x_3 \eand \enot x_2 = x_4 \eand \enot x_3 = x_4)$
%\item[n] $\exists x_1\ldots\exists x_n(Fx_1 \eand\ldots\eand Fx_n \eand \enot x_1 = x_2 \eand\ldots\eand \enot x_{n-1} = x_n)$ 
%\end{ekey}
%
%\subsection*{There are at most \blank\ Fs.}
%\label{summary.atmost}
%
%One way to say `there are at most $n$ Fs' is to put a negation sign in front of the symbolisation for `there are at least $n+1$ Fs'. Equivalently, we can offer:
%\begin{ekey}
%\item[\text{one}] $\forall x_1\forall x_2\bigl[(Fx_1 \eand Fx_2) \eif x_1=x_2\bigr]$
%\item[\text{two}] $\forall x_1\forall x_2\forall x_3\bigl[(Fx_1 \eand Fx_2 \eand Fx_3) \eif (x_1=x_2 \eor x_1=x_3 \eor x_2=x_3)\bigr]$
%\item[\text{three}] $\forall x_1\forall x_2\forall x_3\forall x_4\bigl[(Fx_1 \eand Fx_2 \eand Fx_3 \eand Fx_4) \eif \phantom{.}$\\
%\phantom{$\exists x_1 \exists x_2$}$(x_1=x_2 \eor x_1=x_3 \eor x_1=x_4 \eor x_2=x_3 \eor x_2=x_4 \eor x_3=x_4)\bigr]$
%\item[n]$\forall x_1\ldots\forall x_{n+1}
%\bigl[(Fx_1\eand \ldots \eand Fx_{n+1}) \eif (x_1=x_2 \eor \ldots \eor x_n=x_{n+1})\bigr]$ 
%\end{ekey}
%
%\subsection*{There are exactly \blank\ Fs.}
%\label{summary.exactly}
%
%One way to say `there are exactly $n$ Fs' is to conjoin two of the symbolizations above and say `there are at least $n$ Fs and there are at most $n$ Fs.' The following equivalent formulae are shorter:
%\begin{ekey}
%\item[\text{zero}] $\forall x\enot Fx$
%\item[\text{one}] $\exists x\bigl[Fx \eand \forall y(Fy \eif x= y)\bigr]$
%\item[\text{two}] $\exists x_1\exists x_2\bigl[Fx_1 \eand Fx_2 \eand \enot x_1 = x_2 \eand \forall y\bigl(Fy \eif (y= x_1 \eor y = x_2)\bigr) \bigr]$
%\item[\text{three}] $\exists x_1\exists x_2\exists x_3\bigl[Fx_1 \eand Fx_2 \eand Fx_3 \eand \enot x_1 =  x_2 \eand \enot  x_1 = x_3 \eand \enot x_2 = x_3 \eand \phantom{.}$\\
%\phantom{$\exists x_1 \exists x_2$}$\forall y\bigl(Fy \eif (y = x_1 \eor y = x_2 \eor y =  x_3)\bigr) \bigr]$
%\item[n] $\exists x_1\ldots\exists x_n\bigl[Fx_1 \eand\ldots\eand Fx_n  \eand \enot x_1 = x_2 \eand\ldots\eand \enot x_{n-1}= x_n \eand \phantom{.}$\\
%\phantom{$\exists x_1\exists x_2$}$\forall y\bigl(Fy \eif (y= x_1 \eor \ldots \eor y= x_n)\bigr)\bigr]$ 
%%\item[one] $\exists x\forall y\bigl[Fx \eand (Fy \eif y = x)\bigr]$
%%\item[two] $\exists x\exists y\forall z\Bigl(Fx \eand Fy \eand \bigl[Fz \eif (z=x \eor z=y)\bigr] \eand x \neq y\Bigr)$
%%\item[three] $\exists x_1\exists x_2\exists x_3\forall y\Bigl(Fx_1 \eand Fx_2 \eand Fx_3 \eand [Fy \eif (y=x_1 \eor y=x_2 \eor y=x_3)] \eand x_1 \neq x_2 \eand x_1 \neq x_3 \eand x_2 \neq x_3\Bigr)$
%%\item[n] $\exists x_1\cdots\exists x_n\forall y\Bigl(Fx_1 \eand\cdots\eand Fx_n \eand \bigl[Fy \eif (y=x_1 \eor \cdots \eor y=x_n)\bigr] \eand x_1 \neq x_2 \eand\cdots\eand x_{n-1}\neq x_n\Bigr)$ 
%\end{ekey}
%



\section*{Deduction Rules for TFL}
\renewenvironment{proof}
	{\noindent\par\noindent\small$\begin{nd}}
	{\end{nd}$\noindent\normalsize\ignorespacesafterend}

%{\LARGE \textbf{Basic Rules of Proof}}
\begin{multicols}{2}

\noindent\textbf{Conjunction Introduction}

\begin{proof}
	\have[m]{a}{\meta{A}}
	\have[n]{b}{\meta{B}}
	\have[\ ]{c}{\meta{A}\eand\meta{B}} \ai{a, b}
\end{proof}


\vspace{1em}\noindent\textbf{Conjunction Elimination}

\begin{proof}
	\have[m]{ab}{\meta{A}\eand\meta{B}}
\\	\have[\ ]{a}{\meta{A}} \ae{ab}

	\have[m]{ab}{\meta{A}\eand\meta{B}}
\\	\have[\ ]{b}{\meta{B}} \ae{ab}
\end{proof}

\vspace{1em}\noindent\textbf{Conditional Introduction}

\begin{proof}
	\open
	\hypo[i]{a}{\meta{A}}
	\have[j]{b}{\meta{B}}
	\close
	\have[\ ]{ab}{\meta{A}\eif\meta{B}}\ci{a-b}
\end{proof}

\columnbreak 
\vspace{1em}\noindent\textbf{Conditional Elimination}

\begin{proof}	
	\have[m]{ab}{\meta{A}\eif\meta{B}}
	\\	\have[n]{a}{\meta{A}}
	\have[\ ]{b}{\meta{B}} \ce{ab,a}
\end{proof}

\vspace{1em}\noindent\textbf{Biconditional Introduction}

\begin{proof}
	\open
		\hypo[i]{a1}{\meta{A}} 
		\have[j]{b1}{\meta{B}}
	\close
	\open
		\hypo[k]{b2}{\meta{B}}
		\have[l]{a2}{\meta{A}}
	\close
	\have[\ ]{ab}{\meta{A}\eiff\meta{B}}\bi{a1-b1,b2-a2}
\end{proof}

\vspace{1em}\noindent\textbf{Biconditional Elimination}
\begin{proof}
	\have[m]{ab}{\meta{A}\eiff\meta{B}}
\\	\have[n]{a}{\meta{A}}
	\have[\ ]{b}{\meta{B}} \be{ab,a}
\end{proof}

\end{multicols}

\newpage
\begin{multicols}{2}
\begin{proof}
	\have[m]{ab}{\meta{A}\eiff\meta{B}}
\\	\have[n]{a}{\meta{B}}
	\have[\ ]{b}{\meta{A}} \be{ab,a}
\end{proof}


\vspace{1em}\noindent\textbf{Disjunction Introduction}

\begin{proof}
	\have[m]{a}{\meta{A}}
	\have[\ ]{ab}{\meta{A}\eor\meta{B}}\oi{a}

	\have[m]{a}{\meta{A}}
\\	\have[\ ]{ba}{\meta{B}\eor\meta{A}}\oi{a}
\end{proof}

\vspace{1em}\noindent\textbf{Disjunction Elimination}

\begin{proof}
	\have[m]{ab}{\meta{A}\eor\meta{B}}
\\	\open
		\hypo[i]{a}{\meta{A}}
		\have[j]{c1}{\meta{C}}
	\close
	\open
		\hypo[k]{b}{\meta{B}}
		\have[l]{c2}{\meta{C}}
	\close
	\have[\ ]{c}{\meta{C}} \oe{ab,a-c1, b-c2}
\end{proof}

\vspace{1em}\noindent\textbf{Negation Introduction}

\begin{proof}
	\open
	\hypo[m]{a}{\meta{A}}
	\have[n]{nb}{\ered}
	\close
	\have[\ ]{na}{\enot\meta{A}}\ni{a-nb}
\end{proof}


\vspace{1em}\noindent\textbf{Negation Elimination}

\begin{proof}
	\have[m]{a}{\meta{A}}
	\have[n]{na}{\enot\meta{A}}
	\have[ ]{bot}{\ered}\ri{a, na}
\end{proof}

\vspace{1em}\noindent\textbf{Indirect Proof}

\begin{proof}
	\open
	\hypo[m]{a}{\enot \meta{A}}
	\have[n]{nb}{\ered}
	\close
	\have[\ ]{na}{\meta{A}}\ip{a-nb}
\end{proof}

\end{multicols}

\vspace{1em}
\begin{center}
\textbf{\Large Derived Rules for TFL}
\end{center}


\begin{center}
\begin{tabular}{l  r}
\textbf{Sequent:}                    &       \textbf{Derived Rule:} \\ \hline
$\meta{A} \eor \meta{B},  \enot\meta{B} \proves \meta{A}$ & DS\\
$\meta{A} \eor \meta{B},  \enot\meta{A}\proves \meta{B}$    &      DS  \\
$\meta{A} \rightarrow \meta{B},  \enot\meta{B} \proves \enot \meta{A}$   &                       MT  \\
$\meta{A} \proves\meta{B} \rightarrow \meta{A}$  &              PMI  \\
$\enot\meta{A}\proves\meta{A}\rightarrow \meta{B}$  & PMI\\
$\meta{A} \rightarrow\meta{B} \pequiv \enot\meta{A}\eor \meta{B}$  &                   Imp   \\
$\enot (\meta{A} \rightarrow \meta{B}) \pequiv \meta{A} \eand \enot \meta{B}$  &               NegImp  \\
$\enot (\meta{A} \eand \meta{B})  \pequiv \enot\meta{A}\eor \enot \meta{B}$  &                DeM  \\
$\enot (\meta{A} \eor \meta{B}) \pequiv \enot\meta{A}\ \eand \ \enot \meta{B}$   &               DeM  \\
$\meta{A} \pequiv \enot \enot \meta{A}$     &                              DN  \\
$\meta{A} @ \meta{B}  \proves  \meta{B} @ \meta{A}$ &                          Com \\
$\bot \proves \meta{A}$ & EX \\
 $\proves \meta{A}\eor \enot \meta{A}$ &                                                    LEM \\ 
\end{tabular}
\end{center}


\newpage

\section*{Deduction Rules for FOL}

\begin{multicols}{2}

\vspace{1em}\noindent\textbf{Universal Elimination}

\begin{proof}
	\have[m]{a}{\forall \meta{v}\meta{A}(\ldots \meta{v} \ldots)}
	\have[\ ]{c}{\meta{A}(\ldots \meta{c} \ldots)} \Ae{a}
\end{proof}

\vspace{1em}\noindent\textbf{Universal Introduction}

\begin{proof}
	\open
	\hypo[m]{1}{\meta{c}} \fl{}
	\have[\ ]{2}{\vdots}
	\have[n]{3}{\meta{A}(\ldots \meta{c} \ldots)}
	\close
	\have[\ ]{4}{\forall \meta{v}\meta{A}(\ldots \meta{v} \ldots)} \Ai{1-3}
\end{proof}

\noindent The Flag-ed name \meta{c} may not occur outside the subproof.

\vspace{1em}\noindent\textbf{Existential Introduction}

\begin{proof}
	\have[m]{a}{\meta{A}(\ldots \meta{c} \ldots)}
	\have[\ ]{c}{\exists \meta{x}\meta{A}(\ldots \meta{x} \ldots)} \Ei{a}
\end{proof}

\vspace{1em}\noindent\textbf{Existential Elimination}


\begin{proof}
	\have[m]{a}{\exists \meta{x}\meta{A}(\ldots \meta{x} \ldots)}
	\open	
		\hypo[i]{b}{\meta{A}(\ldots \meta{c} \ldots)} \by{Flag \meta{c}}{}
		\have[j]{c}{\meta{B}}
	\close
	\have[\ ]{d}{\meta{B}} \Ee{a,b-c}
\end{proof}

\noindent The Flag-ed name \meta{c} may not occur outside the subproof.

\end{multicols}


\vspace{1em}
\begin{multicols}{2}

\vspace{1em}\noindent\textbf{Identity Elimination}


\begin{proof}
	\have[m]{e}{\meta{a}=\meta{b}}
	\have[n]{a}{\meta{A}(\ldots \meta{a} \ldots \meta{a}\ldots)}
	\have[\ ]{ea1}{\meta{A}(\ldots \meta{b} \ldots \meta{a}\ldots)} \by{=E}{e,a}
\end{proof}



\begin{proof}
	\have[m]{e}{\meta{a}=\meta{b}}
	\have[n]{a}{\meta{A}(\ldots \meta{b} \ldots \meta{b}\ldots)}
	\have[\ ]{ea2}{\meta{A}(\ldots \meta{a} \ldots \meta{b}\ldots)} \by{=E}{e,a}
\end{proof}

\columnbreak 


\vspace{1em}\noindent\textbf{Identity Introduction}

\begin{proof}
	\have[\ \,\,\,]{x}{\meta{c}=\meta{c}} \by{=I}{}
\end{proof}

\end{multicols}

%\section{Derived rules for FOL}
%\begin{multicols}{2}
%\begin{proof}
%	\have[m]{ab}{\forall \meta{x}\enot \meta{A}}
%	\have[\ ]{ac}{\enot \exists \meta{x} \meta{A}}\cq{ab}
%
%	\have[m]{ab}{\enot \exists \meta{x}  \meta{A}}
%\\	\have[\ ]{ac}{\forall \meta{x}\enot\meta{A}}\cq{ab}
%\end{proof}
%\begin{proof}
%	\have[m]{ab}{\exists \meta{x}\enot\meta{A}}
%	\have[\ ]{ac}{\enot \forall \meta{x} \meta{A}}\cq{ab}
%
%	\have[m]{ab}{\enot \forall \meta{x}  \meta{A}}
%\\	\have[\ ]{ac}{\exists \meta{x}\enot \meta{A}}\cq{ab}
%\end{proof}
%\end{multicols}
%




